 \hypertarget{explain_abduction}{}\section{\-What is Abductive Inference?}\label{explain_abduction}
\-Explain is the component of \-Mistral that performs abductive inference (or abduction). \-Given a formula $ \phi_1 $ and another formula $ \phi_2 $, abduction is the problem of finding an explanatory hypothesis $ \psi $ such that\-: $ \phi_1 \land \psi \models \phi_2 $ and $ \phi_1 \land \psi $ is satisfiable. \-In contrast to logical deduction which reaches valid conclusions from a set of premises, abduction is used to infer likely premises for a given conclusion. \-In general, there are many solutions to an abductive inference problem, but typically, concise and general solutions are more desirable than others. \-The abductive solutions computed by \-Explain are guaranteed to contain a minimum number of variables among all other abductive solutions. \-Furthermore, among other solutions that contain the same set of variables, the solutions computed by \-Explain are as logically weak as possible.\hypertarget{explain_use}{}\section{\-Using Explain}\label{explain_use}
\-To perform abductive inference, use the abduce functions in \hyperlink{Constraint_8h_source}{\-Constraint.\-h}. \-Here is an example illustrating a typical usage of abduction\-:

\begin{DoxyVerb}
  Term* x = VariableTerm::make("x");
  Term* y = VariableTerm::make("y");
  Term* z = VariableTerm::make("z");

  Constraint c1(x, ConstantTerm::make(0), ATOM_LEQ);
  Constraint c2(y, ConstantTerm::make(1), ATOM_GT);
  Constraint premises = c1 & c2;

  map<Term*, long int> elems;
  elems[x] = 2;
  elems[y] = -1;
  elems[z] = 3;
  Term* t = ArithmeticTerm::make(elems);
  Constraint conclusion(t, ConstantTerm::make(10), ATOM_LEQ);

  Constraint explanation = conclusion.abduce(premises);
  cout << "Explanation: " << explanation << endl;
   \end{DoxyVerb}


\-Here, constraint c1 represents the formula x$<$=0, and c2 represents the formula y $>$ 1. \-Hence, the constraint premises represents x$<$=0 \& y $>$ 1. \-The constraint conclusion stands for the formula 2x -\/ y +3z $<$=10. \-Here, we use the abduce function to find a formula called \char`\"{}explanation\char`\"{} such that premises and explanation together imply the conclusion and the premises and explanation are consistent with each other. \-For this example, the solution that is computed by \-Explain (and printed out at the last line) is x$<$=4.

\-Explain also supports computing abductive solutions that must be consistent with a set of given formulas. \-For example, suppose that we want the abductive solution to be consistent with z $>$ 1 as well as x $<$ 2. \-In this case, we can create a set of consistency requirements and pass it as an argument to abduce. \-The following code snippet illustrates this functionality\-:

\begin{DoxyVerb}
Constraint c3(z, ConstantTerm::make(1), ATOM_GT);
Constraint c4(x, ConstantTerm::make(2), ATOM_LT);
set<Constraint> reqs;
reqs.insert(c3);
reqs.insert(c4);
explanation = conclusion.abduce(premises, reqs);
\end{DoxyVerb}


\-In this case, \-Explain will ensure that the computed abductive solution is consistent with every formula in the set reqs.

\-Finally, in some cases, we might prefer abductive solutions that contain certain variables over others. \-For instance, in our earlier example, we might want abductive solutions containing variables z,y over solutions that contain x. \-For this reason, \-Explain allows specifying costs for each variable. \-Here is an example illustrating this functionality\-:

\begin{DoxyVerb}
map<Term*, int> costs;
costs[x] = 5;
costs[y] = 1;
costs[z] = 1;
explanation = conclusion.abduce(premises, reqs, costs);
\end{DoxyVerb}


\-In this case, since the cost of x is higher than the sum of the costs of y,z \-Explain will return solutions that contain both y and z rather than solutions that contain only x. 